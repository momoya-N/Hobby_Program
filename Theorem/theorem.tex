\documentclass[twocolumn,landscape]{ltjsarticle}
\setlength{\columnseprule}{0.2pt}
\usepackage[top=30truemm,bottom=30truemm,left=20truemm,right=20truemm]{geometry}
\usepackage{physics}
\begin{document}

\title{フェーズフィールドメモ}
\maketitle

\section{支配方程式(Allen-Cahn 方程式)}
  \begin{equation}
    \begin{split}
      \tau\pdv{\phi}{t}&=\epsilon^2\laplacian\phi-f'(\phi)\\
      \\
      f(\phi)&=\frac{a^2}{2}g(\phi)+f_S h(\phi)+f_L(1-h(\phi))\\
      g(\phi)&=\phi^2(1-\phi)^2\\
      h(\phi)&=\phi^2(3-2\phi)
    \end{split}    
  \end{equation}
  以上より、
  \begin{equation}
    \tau\pdv{\phi}{t}=\epsilon^2\laplacian\phi-+2a^2\phi(1-\phi)\qty{\phi-\frac{1}{2}+\frac{3(f_L-f_S)}{a^2}}
  \end{equation}
  離散化すると、
  \begin{equation}
    \begin{split}
      \phi(t+dt,\vb*{x})=&\phi(t,\vb*{x})+\frac{dt \epsilon^2}{\tau h^2}\left\{\phi(t,x+h,y)+\phi(t,x-h,y)+\phi(t,x,y+h)+\right.\\
      &\left.\phi(t,x,y-h)-4\phi(t,x,y)\right\}+2a^2\phi(1-\phi)\frac{dt}{\tau}\qty{\phi-\frac{1}{2}+\frac{3(f_L-f_S)}{a^2}}      
    \end{split}
  \end{equation}
\end{document}

